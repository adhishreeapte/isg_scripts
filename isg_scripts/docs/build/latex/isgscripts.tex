%% Generated by Sphinx.
\def\sphinxdocclass{report}
\documentclass[letterpaper,10pt,english]{sphinxmanual}
\ifdefined\pdfpxdimen
   \let\sphinxpxdimen\pdfpxdimen\else\newdimen\sphinxpxdimen
\fi \sphinxpxdimen=.75bp\relax
\ifdefined\pdfimageresolution
    \pdfimageresolution= \numexpr \dimexpr1in\relax/\sphinxpxdimen\relax
\fi
%% let collapsible pdf bookmarks panel have high depth per default
\PassOptionsToPackage{bookmarksdepth=5}{hyperref}

\PassOptionsToPackage{warn}{textcomp}
\usepackage[utf8]{inputenc}
\ifdefined\DeclareUnicodeCharacter
% support both utf8 and utf8x syntaxes
  \ifdefined\DeclareUnicodeCharacterAsOptional
    \def\sphinxDUC#1{\DeclareUnicodeCharacter{"#1}}
  \else
    \let\sphinxDUC\DeclareUnicodeCharacter
  \fi
  \sphinxDUC{00A0}{\nobreakspace}
  \sphinxDUC{2500}{\sphinxunichar{2500}}
  \sphinxDUC{2502}{\sphinxunichar{2502}}
  \sphinxDUC{2514}{\sphinxunichar{2514}}
  \sphinxDUC{251C}{\sphinxunichar{251C}}
  \sphinxDUC{2572}{\textbackslash}
\fi
\usepackage{cmap}
\usepackage[T1]{fontenc}
\usepackage{amsmath,amssymb,amstext}
\usepackage{babel}



\usepackage{tgtermes}
\usepackage{tgheros}
\renewcommand{\ttdefault}{txtt}



\usepackage[Bjarne]{fncychap}
\usepackage{sphinx}

\fvset{fontsize=auto}
\usepackage{geometry}


% Include hyperref last.
\usepackage{hyperref}
% Fix anchor placement for figures with captions.
\usepackage{hypcap}% it must be loaded after hyperref.
% Set up styles of URL: it should be placed after hyperref.
\urlstyle{same}

\addto\captionsenglish{\renewcommand{\contentsname}{Contents:}}

\usepackage{sphinxmessages}
\setcounter{tocdepth}{1}



\title{ISG Scripts Documentation}
\date{Feb 15, 2023}
\release{v2.0.0}
\author{Sedemac R\&D}
\newcommand{\sphinxlogo}{\vbox{}}
\renewcommand{\releasename}{Release}
\makeindex
\begin{document}

\ifdefined\shorthandoff
  \ifnum\catcode`\=\string=\active\shorthandoff{=}\fi
  \ifnum\catcode`\"=\active\shorthandoff{"}\fi
\fi

\pagestyle{empty}
\sphinxmaketitle
\pagestyle{plain}
\sphinxtableofcontents
\pagestyle{normal}
\phantomsection\label{\detokenize{index::doc}}


\sphinxAtStartPar
This repository houses scripts used to analyse data and generate plots for customer demos of ISG applications.
PCAN traces are taken when cranking on vehicle.

\sphinxAtStartPar
Modules for PCAN trace files:
\begin{itemize}
\item {} 
\sphinxAtStartPar
firetrace

\item {} 
\sphinxAtStartPar
deadtrace

\item {} 
\sphinxAtStartPar
assist

\end{itemize}

\sphinxAtStartPar
Modules for Picoscope \sphinxcode{\sphinxupquote{.csv}} files:
\begin{itemize}
\item {} 
\sphinxAtStartPar
speedtime module

\end{itemize}

\sphinxstepscope


\chapter{Firetrace}
\label{\detokenize{firetrace:firetrace}}\label{\detokenize{firetrace::doc}}
\sphinxAtStartPar
Firetraces are traces of engine starts taken with sparkplug.

\begin{sphinxVerbatim}[commandchars=\\\{\}]
\PYG{p}{\PYGZob{}}
\PYG{l+s+s2}{\PYGZdq{}}\PYG{l+s+s2}{sym\PYGZus{}file}\PYG{l+s+s2}{\PYGZdq{}} \PYG{p}{:} \PYG{l+s+s2}{\PYGZdq{}}\PYG{l+s+s2}{Symbol\PYGZus{}file\PYGZus{}isg\PYGZus{}assist\PYGZus{}codebase.sym}\PYG{l+s+s2}{\PYGZdq{}}\PYG{p}{,}
\PYG{l+s+s2}{\PYGZdq{}}\PYG{l+s+s2}{trace\PYGZus{}file}\PYG{l+s+s2}{\PYGZdq{}} \PYG{p}{:} \PYG{l+s+s2}{\PYGZdq{}}\PYG{l+s+s2}{v13\PYGZus{}working\PYGZus{}more\PYGZus{}clean\PYGZus{}cranks.trc}\PYG{l+s+s2}{\PYGZdq{}}\PYG{p}{,}
\PYG{l+s+s2}{\PYGZdq{}}\PYG{l+s+s2}{vertical\PYGZus{}speed\PYGZus{}jump}\PYG{l+s+s2}{\PYGZdq{}} \PYG{p}{:} \PYG{l+m+mi}{550}\PYG{p}{,}
\PYG{l+s+s2}{\PYGZdq{}}\PYG{l+s+s2}{jump\PYGZus{}time\PYGZus{}duration}\PYG{l+s+s2}{\PYGZdq{}} \PYG{p}{:} \PYG{l+m+mf}{0.03}\PYG{p}{,}
\PYG{l+s+s2}{\PYGZdq{}}\PYG{l+s+s2}{idling\PYGZus{}speed}\PYG{l+s+s2}{\PYGZdq{}} \PYG{p}{:} \PYG{l+m+mi}{1500}\PYG{p}{,}
\PYG{l+s+s2}{\PYGZdq{}}\PYG{l+s+s2}{m\PYGZus{}speed}\PYG{l+s+s2}{\PYGZdq{}} \PYG{p}{:} \PYG{l+s+s2}{\PYGZdq{}}\PYG{l+s+s2}{Bemf\PYGZus{}Speed\PYGZus{}RPM}\PYG{l+s+s2}{\PYGZdq{}}\PYG{p}{,}
\PYG{l+s+s2}{\PYGZdq{}}\PYG{l+s+s2}{operation\PYGZus{}mode}\PYG{l+s+s2}{\PYGZdq{}} \PYG{p}{:} \PYG{l+s+s2}{\PYGZdq{}}\PYG{l+s+s2}{MEAS\PYGZus{}OPMODE}\PYG{l+s+s2}{\PYGZdq{}}\PYG{p}{,}
\PYG{l+s+s2}{\PYGZdq{}}\PYG{l+s+s2}{battery\PYGZus{}current}\PYG{l+s+s2}{\PYGZdq{}} \PYG{p}{:} \PYG{l+s+s2}{\PYGZdq{}}\PYG{l+s+s2}{IDC\PYGZus{}Estimated}\PYG{l+s+s2}{\PYGZdq{}}\PYG{p}{,}
\PYG{l+s+s2}{\PYGZdq{}}\PYG{l+s+s2}{battery\PYGZus{}voltage}\PYG{l+s+s2}{\PYGZdq{}} \PYG{p}{:} \PYG{l+s+s2}{\PYGZdq{}}\PYG{l+s+s2}{Vbat}\PYG{l+s+s2}{\PYGZdq{}}\PYG{p}{,}
\PYG{l+s+s2}{\PYGZdq{}}\PYG{l+s+s2}{u\PYGZus{}theta}\PYG{l+s+s2}{\PYGZdq{}} \PYG{p}{:} \PYG{l+s+s2}{\PYGZdq{}}\PYG{l+s+s2}{MEAS\PYGZus{}UTHETA}\PYG{l+s+s2}{\PYGZdq{}}
\PYG{p}{\PYGZcb{}}
\end{sphinxVerbatim}

\sphinxAtStartPar
Example set of config, trace, and \sphinxcode{\sphinxupquote{.sym}} files are \sphinxhref{\_static/files/firetrace/fireconfig.json}{config.json}, \sphinxhref{\_static/files/firetrace/firetrace.trc}{trace.trc} and \sphinxhref{\_static/files/firetrace/Symbol\_file\_isg\_assist\_codebase.sym}{symbol.sym}.
Command below is used to analyse firetrace taken from PCAN.
\sphinxSetupCaptionForVerbatim{Command}
\def\sphinxLiteralBlockLabel{\label{\detokenize{firetrace:id1}}}
\begin{sphinxVerbatim}[commandchars=\\\{\}]
isg.firetrace\PYG{+w}{ }\PYGZhy{}\PYGZhy{}config\PYG{+w}{ }config.json
\end{sphinxVerbatim}

\sphinxAtStartPar
Description of config.json file:
\begin{itemize}
\item {} 
\sphinxAtStartPar
“sym\_file” : Name of PCAN \sphinxcode{\sphinxupquote{.sym}} file

\item {} 
\sphinxAtStartPar
“trace\_file” : Name of \sphinxcode{\sphinxupquote{.trc}} file

\item {} 
\sphinxAtStartPar
“speed\_jump” : Speed jump in RPM at fire point. Motor\sphinxhyphen{}engine specific.

\item {} 
\sphinxAtStartPar
“jump\_time\_duration” : Time in seconds required to achieve “vertical\_speed\_jump” at fire point.

\item {} 
\sphinxAtStartPar
“m\_speed” : Speed variable name in \sphinxcode{\sphinxupquote{.sym}} file

\item {} 
\sphinxAtStartPar
“operation\_mode” : Op\_mode variable name in \sphinxcode{\sphinxupquote{.sym}} file

\item {} 
\sphinxAtStartPar
“battery\_current” : Ibat variable name in \sphinxcode{\sphinxupquote{.sym}} file

\item {} 
\sphinxAtStartPar
“battery\_voltage” : Vbat variable name in \sphinxcode{\sphinxupquote{.sym}} file

\item {} 
\sphinxAtStartPar
“u\_theta” : U\_theta variable name in \sphinxcode{\sphinxupquote{.sym}} file

\end{itemize}

\sphinxAtStartPar
Description of script:
\begin{itemize}
\item {} 
\sphinxAtStartPar
Successful cranks detection based on op\_\_mode

\item {} 
\sphinxAtStartPar
For each successful crank
\begin{itemize}
\item {} 
\sphinxAtStartPar
Calculate reverse bang time based on utheta

\item {} 
\sphinxAtStartPar
Engine\sphinxhyphen{}fire\sphinxhyphen{}time based on jump in speed

\end{itemize}

\item {} 
\sphinxAtStartPar
Jump detection requires engine\sphinxhyphen{}motor specific parameter \sphinxcode{\sphinxupquote{speed\_jump}}. 550 RPM in 0.03 seconds is for NTorq.

\item {} 
\sphinxAtStartPar
The value of \sphinxcode{\sphinxupquote{jump\_time\_duration}} should be increased if m\_speed data transmits are sparse.

\end{itemize}

\sphinxAtStartPar
Figure outputs are saved in folder named \sphinxcode{\sphinxupquote{isg\_plots}} created at terminal location.
A sample subset of the figures is shown below.

\noindent\sphinxincludegraphics{{firetrace_singlecrank}.png}

\noindent\sphinxincludegraphics{{firetrace_banghist}.png}

\noindent\sphinxincludegraphics{{firetrace_firehist}.png}

\noindent\sphinxincludegraphics{{firetrace_en95hist}.png}

\sphinxAtStartPar
Text\sphinxhyphen{}output is written in \sphinxcode{\sphinxupquote{firetrace\_output.csv}} file at terminal location.

\sphinxstepscope


\chapter{Deadtrace}
\label{\detokenize{deadtrace:deadtrace}}\label{\detokenize{deadtrace::doc}}
\sphinxAtStartPar
Deadtraces are taken with sparkplug removed.
Example \sphinxcode{\sphinxupquote{config.json}} file is shown below:

\begin{sphinxVerbatim}[commandchars=\\\{\}]
\PYG{o}{\PYGZob{}}
\PYG{l+s+s2}{\PYGZdq{}sym\PYGZus{}file\PYGZdq{}}\PYG{+w}{ }:\PYG{+w}{ }\PYG{l+s+s2}{\PYGZdq{}Symbol\PYGZus{}file\PYGZus{}isg\PYGZus{}assist\PYGZus{}codebase.sym\PYGZdq{}},
\PYG{l+s+s2}{\PYGZdq{}trace\PYGZus{}file\PYGZdq{}}\PYG{+w}{ }:\PYG{+w}{ }\PYG{l+s+s2}{\PYGZdq{}true\PYGZus{}dead\PYGZus{}cranks\PYGZus{}v13.trc\PYGZdq{}},
\PYG{l+s+s2}{\PYGZdq{}m\PYGZus{}speed\PYGZdq{}}\PYG{+w}{ }:\PYG{+w}{ }\PYG{l+s+s2}{\PYGZdq{}Bemf\PYGZus{}Speed\PYGZus{}RPM\PYGZdq{}},
\PYG{l+s+s2}{\PYGZdq{}operation\PYGZus{}mode\PYGZdq{}}\PYG{+w}{ }:\PYG{+w}{ }\PYG{l+s+s2}{\PYGZdq{}MEAS\PYGZus{}OPMODE\PYGZdq{}},
\PYG{l+s+s2}{\PYGZdq{}battery\PYGZus{}current\PYGZdq{}}\PYG{+w}{ }:\PYG{+w}{ }\PYG{l+s+s2}{\PYGZdq{}IDC\PYGZus{}Estimated\PYGZdq{}},
\PYG{l+s+s2}{\PYGZdq{}battery\PYGZus{}voltage\PYGZdq{}}\PYG{+w}{ }:\PYG{+w}{ }\PYG{l+s+s2}{\PYGZdq{}Vbat\PYGZdq{}},
\PYG{l+s+s2}{\PYGZdq{}u\PYGZus{}theta\PYGZdq{}}\PYG{+w}{ }:\PYG{+w}{ }\PYG{l+s+s2}{\PYGZdq{}MEAS\PYGZus{}UTHETA\PYGZdq{}},
\PYG{l+s+s2}{\PYGZdq{}ia\PYGZdq{}}\PYG{+w}{ }:\PYG{+w}{ }\PYG{l+s+s2}{\PYGZdq{}IA\PYGZdq{}}
\PYG{o}{\PYGZcb{}}
\end{sphinxVerbatim}

\sphinxAtStartPar
Example set of config, \sphinxcode{\sphinxupquote{trace\_file}}, and \sphinxcode{\sphinxupquote{sym\_file}} files are \sphinxhref{\_static/files/deadtrace/deadconfig.json}{config.json}, \sphinxhref{\_static/files/deadtrace/deadtrace.trc}{trace.trc} and \sphinxhref{\_static/files/Symbol\_file\_isg\_assist\_codebase.sym}{symbol.sym}.

\sphinxAtStartPar
Command below is used to analyse deadtrace taken from PCAN.
\sphinxSetupCaptionForVerbatim{Command}
\def\sphinxLiteralBlockLabel{\label{\detokenize{deadtrace:id1}}}
\begin{sphinxVerbatim}[commandchars=\\\{\}]
isg.deadtrace\PYG{+w}{ }\PYGZhy{}\PYGZhy{}config\PYG{+w}{ }deadconfig.json
\end{sphinxVerbatim}

\sphinxAtStartPar
Description of config.json file:
\begin{itemize}
\item {} 
\sphinxAtStartPar
“sym\_file” : Name of PCAN \sphinxcode{\sphinxupquote{.sym}} file

\item {} 
\sphinxAtStartPar
“trace\_file” : Name of \sphinxcode{\sphinxupquote{trace}} file

\item {} 
\sphinxAtStartPar
“m\_speed” : Speed variable name in \sphinxcode{\sphinxupquote{.sym}} file

\item {} 
\sphinxAtStartPar
“operation\_mode” : Op\_mode variable name in \sphinxcode{\sphinxupquote{.sym}} file

\item {} 
\sphinxAtStartPar
“battery\_current” : Ibat variable name in \sphinxcode{\sphinxupquote{.sym}} file

\item {} 
\sphinxAtStartPar
“battery\_voltage” : Vbat variable name in \sphinxcode{\sphinxupquote{.sym}} file

\item {} 
\sphinxAtStartPar
“u\_theta” : U\_theta variable name in \sphinxcode{\sphinxupquote{.sym}} file

\item {} 
\sphinxAtStartPar
“ia” : Phase current variable name in \sphinxcode{\sphinxupquote{.sym}} file

\end{itemize}

\sphinxAtStartPar
Description of script:
\begin{itemize}
\item {} 
\sphinxAtStartPar
Script calculates reverse bang time based on utheta.

\item {} 
\sphinxAtStartPar
Sci\sphinxhyphen{}py is used to find compression times to isolate thermodynamic cycles.

\item {} 
\sphinxAtStartPar
Statistics of power drawn from the battery, battery\sphinxhyphen{}current, battery\sphinxhyphen{}voltage and energy consumed per cycle are generated.

\item {} 
\sphinxAtStartPar
At every compression, current drawn from battery is maximum and battery voltage dips to a minimum.

\item {} 
\sphinxAtStartPar
Thus, statistics include maximum, mean of ibat and minimum, mean of vbat.

\end{itemize}

\sphinxAtStartPar
Figure output is placed in folder named \sphinxcode{\sphinxupquote{isg\_plots}} created at the location of terminal initiation.
A sample subset of the output\sphinxhyphen{}figures is shown below.

\noindent\sphinxincludegraphics{{deadtrace_vbat}.png}

\noindent\sphinxincludegraphics{{deadtrace_ibat}.png}

\noindent\sphinxincludegraphics{{deadtrace_power}.png}

\noindent\sphinxincludegraphics{{deadtrace_energy}.png}

\sphinxAtStartPar
Text\sphinxhyphen{}output include line rms maximum and average written in \sphinxcode{\sphinxupquote{deadtrace\_output.csv}} file at terminal location.

\sphinxstepscope


\chapter{Assist}
\label{\detokenize{assist:assist}}\label{\detokenize{assist::doc}}
\sphinxAtStartPar
The following command is used to analyse assist and charging efficiency taken from PCAN.
The trace files contains charging or assist.

\begin{sphinxVerbatim}[commandchars=\\\{\}]
\PYG{p}{\PYGZob{}}
    \PYG{l+s+s2}{\PYGZdq{}}\PYG{l+s+s2}{begin\PYGZus{}time}\PYG{l+s+s2}{\PYGZdq{}} \PYG{p}{:} \PYG{p}{[}\PYG{l+m+mi}{11}\PYG{p}{,} \PYG{l+m+mi}{54}\PYG{p}{,} \PYG{l+m+mi}{24}\PYG{p}{]}\PYG{p}{,}
    \PYG{l+s+s2}{\PYGZdq{}}\PYG{l+s+s2}{end\PYGZus{}time}\PYG{l+s+s2}{\PYGZdq{}} \PYG{p}{:} \PYG{p}{[}\PYG{l+m+mi}{11}\PYG{p}{,} \PYG{l+m+mi}{54}\PYG{p}{,} \PYG{l+m+mi}{25}\PYG{p}{]}\PYG{p}{,}
    \PYG{l+s+s2}{\PYGZdq{}}\PYG{l+s+s2}{Rs}\PYG{l+s+s2}{\PYGZdq{}} \PYG{p}{:} \PYG{l+m+mf}{33.5}\PYG{p}{,}
    \PYG{l+s+s2}{\PYGZdq{}}\PYG{l+s+s2}{sym\PYGZus{}file}\PYG{l+s+s2}{\PYGZdq{}} \PYG{p}{:} \PYG{l+s+s2}{\PYGZdq{}}\PYG{l+s+s2}{Symbol\PYGZus{}file\PYGZus{}isg\PYGZus{}assist\PYGZus{}codebase.sym}\PYG{l+s+s2}{\PYGZdq{}}\PYG{p}{,}
    \PYG{l+s+s2}{\PYGZdq{}}\PYG{l+s+s2}{trace\PYGZus{}file}\PYG{l+s+s2}{\PYGZdq{}} \PYG{p}{:} \PYG{l+s+s2}{\PYGZdq{}}\PYG{l+s+s2}{assist\PYGZus{}test\PYGZus{}24nov.trc}\PYG{l+s+s2}{\PYGZdq{}}\PYG{p}{,}
    \PYG{l+s+s2}{\PYGZdq{}}\PYG{l+s+s2}{battery\PYGZus{}current}\PYG{l+s+s2}{\PYGZdq{}} \PYG{p}{:} \PYG{l+s+s2}{\PYGZdq{}}\PYG{l+s+s2}{IDC\PYGZus{}Estimated}\PYG{l+s+s2}{\PYGZdq{}}\PYG{p}{,}
    \PYG{l+s+s2}{\PYGZdq{}}\PYG{l+s+s2}{battery\PYGZus{}voltage}\PYG{l+s+s2}{\PYGZdq{}} \PYG{p}{:} \PYG{l+s+s2}{\PYGZdq{}}\PYG{l+s+s2}{Vbat}\PYG{l+s+s2}{\PYGZdq{}}\PYG{p}{,}
    \PYG{l+s+s2}{\PYGZdq{}}\PYG{l+s+s2}{assist\PYGZus{}state}\PYG{l+s+s2}{\PYGZdq{}} \PYG{p}{:} \PYG{l+s+s2}{\PYGZdq{}}\PYG{l+s+s2}{Assist\PYGZus{}State}\PYG{l+s+s2}{\PYGZdq{}}\PYG{p}{,}
    \PYG{l+s+s2}{\PYGZdq{}}\PYG{l+s+s2}{ia}\PYG{l+s+s2}{\PYGZdq{}} \PYG{p}{:} \PYG{l+s+s2}{\PYGZdq{}}\PYG{l+s+s2}{IA}\PYG{l+s+s2}{\PYGZdq{}}\PYG{p}{,}
    \PYG{l+s+s2}{\PYGZdq{}}\PYG{l+s+s2}{ib}\PYG{l+s+s2}{\PYGZdq{}} \PYG{p}{:} \PYG{l+s+s2}{\PYGZdq{}}\PYG{l+s+s2}{IB}\PYG{l+s+s2}{\PYGZdq{}}\PYG{p}{,}
    \PYG{l+s+s2}{\PYGZdq{}}\PYG{l+s+s2}{ic}\PYG{l+s+s2}{\PYGZdq{}} \PYG{p}{:} \PYG{l+s+s2}{\PYGZdq{}}\PYG{l+s+s2}{IC}\PYG{l+s+s2}{\PYGZdq{}}\PYG{p}{,}
    \PYG{l+s+s2}{\PYGZdq{}}\PYG{l+s+s2}{charge\PYGZus{}state}\PYG{l+s+s2}{\PYGZdq{}} \PYG{p}{:} \PYG{l+s+s2}{\PYGZdq{}}\PYG{l+s+s2}{Charging\PYGZus{}State}\PYG{l+s+s2}{\PYGZdq{}}\PYG{p}{,}
    \PYG{l+s+s2}{\PYGZdq{}}\PYG{l+s+s2}{a\PYGZus{}or\PYGZus{}c}\PYG{l+s+s2}{\PYGZdq{}} \PYG{p}{:} \PYG{l+s+s2}{\PYGZdq{}}\PYG{l+s+s2}{a}\PYG{l+s+s2}{\PYGZdq{}}
\PYG{p}{\PYGZcb{}}
\end{sphinxVerbatim}

\sphinxAtStartPar
Example set of config, trace, and \sphinxcode{\sphinxupquote{.sym}} files are \sphinxhref{\_static/files/assist/assistconfig.json}{config.json}, \sphinxhref{\_static/files/assist\_test\_24nov.trc}{trace.trc} and \sphinxhref{\_static/files/Symbol\_file\_isg\_assist\_codebase.sym}{symbol.sym}.

\sphinxAtStartPar
Command below is used to analyse trace taken from PCAN.
\sphinxSetupCaptionForVerbatim{Command}
\def\sphinxLiteralBlockLabel{\label{\detokenize{assist:id1}}}
\begin{sphinxVerbatim}[commandchars=\\\{\}]
isg.assist\PYG{+w}{ }\PYGZhy{}\PYGZhy{}config\PYG{+w}{ }assist.json
\end{sphinxVerbatim}

\sphinxAtStartPar
Description of config.json file:
\begin{itemize}
\item {} 
\sphinxAtStartPar
“sym\_file” : Name of PCAN \sphinxcode{\sphinxupquote{.sym}} file

\item {} 
\sphinxAtStartPar
“trace\_file” : Name of \sphinxcode{\sphinxupquote{.trc}} file

\item {} 
\sphinxAtStartPar
“begin\_time” : Speed jump in RPM at fire point. Motor\sphinxhyphen{}engine specific.

\item {} 
\sphinxAtStartPar
“end\_time” : Time in seconds required to achieve “vertical\_speed\_jump” at fire point.

\item {} 
\sphinxAtStartPar
“Rs” : Phase resistance in mOhm of the motor

\item {} 
\sphinxAtStartPar
“a\_or\_c” : Assist “a” or charging mode “c”

\item {} 
\sphinxAtStartPar
“operation\_mode” : Op\_mode variable name in \sphinxcode{\sphinxupquote{.sym}} file

\item {} 
\sphinxAtStartPar
“battery\_current” : Ibat variable name in \sphinxcode{\sphinxupquote{.sym}} file

\item {} 
\sphinxAtStartPar
“battery\_voltage” : Vbat variable name in \sphinxcode{\sphinxupquote{.sym}} file

\item {} 
\sphinxAtStartPar
“assist\_state” : Assist state variable name in \sphinxcode{\sphinxupquote{.sym}} file

\item {} 
\sphinxAtStartPar
“charge\_state” : Charging state variable name in \sphinxcode{\sphinxupquote{.sym}} file

\item {} 
\sphinxAtStartPar
“ia” : Phase current A in \sphinxcode{\sphinxupquote{.sym}} file

\item {} 
\sphinxAtStartPar
“ib” : Phase current B in \sphinxcode{\sphinxupquote{.sym}} file

\item {} 
\sphinxAtStartPar
“ic” : Phase current C in \sphinxcode{\sphinxupquote{.sym}} file

\end{itemize}

\sphinxAtStartPar
Energy from battery:
\begin{itemize}
\item {} 
\sphinxAtStartPar
e\_bat = Vbat.Ibat.t

\end{itemize}

\sphinxAtStartPar
Copper loss:
\begin{itemize}
\item {} 
\sphinxAtStartPar
e\_loss = Rs.(ia\textasciicircum{}2 + ib\textasciicircum{}2 + ic\textasciicircum{}2)t

\end{itemize}

\sphinxAtStartPar
Efficiency:
\begin{itemize}
\item {} 
\sphinxAtStartPar
Charging\sphinxhyphen{}
\begin{itemize}
\item {} 
\sphinxAtStartPar
eta = \sphinxhyphen{}1*e\_bat/(\sphinxhyphen{}1*e\_bat + e\_loss)

\end{itemize}

\item {} 
\sphinxAtStartPar
Assist\sphinxhyphen{}
\begin{itemize}
\item {} 
\sphinxAtStartPar
eta = 1 \sphinxhyphen{} (e\_loss/e\_bat)

\end{itemize}

\end{itemize}

\sphinxAtStartPar
Voltage ripple, efficiency, energy from battery and copper loss values are printed on command line.
Sample output is in figure below :

\noindent\sphinxincludegraphics{{assist_output}.png}

\sphinxstepscope


\chapter{Speedtime}
\label{\detokenize{speedtime:speedtime}}\label{\detokenize{speedtime::doc}}
\sphinxAtStartPar
Command below is used to analyse csv files created from data taken on Picoscope.
\sphinxSetupCaptionForVerbatim{Command}
\def\sphinxLiteralBlockLabel{\label{\detokenize{speedtime:id1}}}
\begin{sphinxVerbatim}[commandchars=\\\{\}]
isg.speedtime\PYG{+w}{ }\PYGZhy{}\PYGZhy{}file\PYGZus{}name\PYG{+w}{ }picodata.csv
\end{sphinxVerbatim}

\sphinxAtStartPar
Example \sphinxcode{\sphinxupquote{.csv}} file is \sphinxhref{\_static/files/speedtime/pico\_data.csv}{pico\_data.csv}.

\sphinxAtStartPar
Output filtered speed is populated in \sphinxcode{\sphinxupquote{output\_speedtime.csv}}.
Plot is saved in folder named “isg\_plots” created at location of terminal.

\noindent\sphinxincludegraphics{{speedtime}.png}


\chapter{Codebase}
\label{\detokenize{index:codebase}}
\sphinxAtStartPar
Download page: \sphinxurl{https://bitbucket.org/sedemac/isg\_scripts/src/master/}


\chapter{Indices and tables}
\label{\detokenize{index:indices-and-tables}}\begin{itemize}
\item {} 
\sphinxAtStartPar
\DUrole{xref,std,std-ref}{genindex}

\item {} 
\sphinxAtStartPar
\DUrole{xref,std,std-ref}{modindex}

\item {} 
\sphinxAtStartPar
\DUrole{xref,std,std-ref}{search}

\end{itemize}



\renewcommand{\indexname}{Index}
\printindex
\end{document}